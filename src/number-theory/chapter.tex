\section{Number Theory}

\subsection{Euler's totient function}
\begin{itemize}
	\item Euler's totient function, also known as \textbf{phi-function} $\phi(n)$ 
	counts the number of integers between 1 and $n$ inclusive, that are
	\textbf{coprime to} $n$.
	\item Properties:
	\begin{itemize}
		\item Divisor sum property: $\sum\limits_{d | n} \phi(d) = n$.
		\item $\phi(n)$ is a \textbf{prime number} when $n$ = 3, 4, 6.
		\item If $p$ is a prime number, then $\phi(p) = p - 1$.
		\item If $p$ is a prime number and $k \geq 1$, then $\phi(p^k) = p^k - p^{k - 1}$.
		\item If $a$ and $b$ are \textbf{coprime}, then $\phi(ab) = \phi(a) \cdot \phi(b)$.
		\item In general, for \textbf{not coprime} $a$ and $b$, with $d = gcd(a, b)$ this equation holds: 
		$\phi(ab) = \phi(a) \cdot \phi(b) \cdot \dfrac{d}{\phi(d)}$.
		\item With $n = p_1^{k_1} \cdot p_2^{k_2} \cdots p_m^{k_m}$:
		\begin{align*}
			\phi(n) &= \phi(p_1^{k_1}) \cdot \phi(p_2^{k_2}) \cdots \phi(p_m^{k_m}) \\ 
			&= n \cdot \left(1 - \dfrac{1}{p_1}\right) \cdot \left(1 - \dfrac{1}{p_2}\right) \cdots \cdot \left(1 - \dfrac{1}{p_m}\right)
		\end{align*}
	\end{itemize}
	\item Application in Euler's theorem:
	\begin{itemize}
		\item If gcd($a, M$) = 1, then:
		\[ a^{\phi(M)} \equiv 1\ (\text{mod}\ M) \Rightarrow a^{n} \equiv a^{n\ \text{mod}\ M}\ (\text{mod}\ M)\]
		\item In general, for arbitrary $a$, $M$ and $n \geq \log_2{M}$:
		\[ a^{n} \equiv a^{\phi(M) + [n\ \text{mod}\ \phi(M)]}\ (\text{mod}\ M)\]
	\end{itemize}
\end{itemize}

\subsection{Mobius function}
\begin{itemize}
	\item For a positive integer $n = p_1^{k_1} \cdot p_2^{k_2} \cdots p_m^{k_m}$:
		\[
		\mu(n) =
		\begin{cases}
			1, & \text{if}\ n = 1\\
			0, & \text{if}\ \exists k_i > 1\\
			(-1)^{m} & \text{otherwise}
		\end{cases}
		\]
	\item Properties:
	\begin{itemize}
		\item $\sum\limits_{d | n} \mu(d) = [n = 1]$.
		\item If $a$ and $b$ are \textbf{coprime}, then $\mu(ab) = \mu(a) \cdot \mu(b)$.
		\item Mobius inversion: let $f$ and $g$ be arithmetic functions:
		\[ g(n) = \sum\limits_{d | n} f(d) \Leftrightarrow f(n) = \sum\limits_{d | n} \mu\left(\dfrac{n}{d}\right) g(d) \].
	\end{itemize}
\end{itemize}

\subsection{Primes}
Approximating the number of primes up to $n$: 
\begin{center}
	\begin{tabular}{| l | l | l | }
		\hline
		$n$ & $\pi(n)$ & $\dfrac{n}{\ln{n} - 1}$ \\
		\hline
		% $1e^2$ & 25 & ??? & ??? \\ 
		% $5e^2$ & 95 & ??? & ??? \\ 
		% $1e^3$ & 168 & 145 & 169 \\ 
		% $5e^3$ & 669 & ??? & ??? \\ 
		% $1e^4$ & 1229 & 1086 & 1218 \\ 
		% $5e^4$ & 5133 & ??? & ??? \\ 
		% $1e^5$ & 9592 & 8686 & 9512 \\ 
		% $5e^5$ & 41538 & ??? & ??? \\ 
		% $1e^6$ & 78498 & 72382 & 78030 \\ 
		% $5e^6$ & 348513 & ??? & ??? \\ 

		$100\ (1e^2)$ & 25 & 28 \\ 
		$500\ (5e^2)$ & 95 & 96 \\ 
		$1000\ (1e^3)$ & 168 & 169 \\ 
		$5000\ (5e^3)$ & 669 & 665 \\ 
		$10000\ (1e^4)$ & 1229 & 1218 \\ 
		$50000\ (5e^4)$ & 5133 & 5092 \\ 
		$100000\ (1e^5)$ & 9592 & 9512 \\ 
		$500000\ (5e^5)$ & 41538 & 41246 \\ 
		$1000000\ (1e^6)$ & 78498 & 78030 \\ 
		$5000000\ (5e^6)$ & 348513 & 346622 \\ 
		\hline
	\end{tabular}
\end{center}
($\pi(n)$ = the number of primes less than or equal to $n$, $\dfrac{n}{\ln{n} - 1}$ is used to approximate $\pi(n)$).

\subsection{Wilson's theorem}
A positive integer $n$ is a prime if and only if:
\[ (n - 1)! \equiv n - 1\ (\text{mod}\ n)\]

\subsection{Zeckendorf’s theorem}
The Zeckendorf's theorem states that every positive integer $n$ can be represented uniquely as a sum of one or more 
distinct non-consecutive Fibonacci numbers. For example:
	\vspace{-0.45cm}
	\begin{align*}
		64 &= 55 + 8 + 1 \\ 
		85 &= 55 + 21 + 8 + 1
	\end{align*}
	\vspace{-0.45cm}
	\includecpp{zeckendorf_theorem.h}

\subsection{Bitwise operation}
\begin{multicols}{2}
\vspace{-\topsep}
\begin{itemize}
  	\setlength{\parskip}{0pt}
 	\setlength{\itemsep}{0pt plus 1pt}
	\item $a + b = (a \oplus b) + 2(a\ \&\ b)$
	\item $a\ |\ b = (a \oplus b) + (a\ \&\ b)$
	\item $a\ \&\ (b \oplus c) = (a\ \&\ b) \oplus (a\ \&\ c)$
	\item $a\ |\ (b\ \&\ c) = (a\ |\ b)\ \&\ (a\ |\ c)$
	\item $a\ \&\ (b\ |\ c) = (a\ \&\ b)\ |\ (a\ \&\ c)$
  	\item $a\ |\ (a\ \&\ b) = a$
	\item $a\ \&\ (a\ |\ b) = a$
	\item $n = 2 ^ k \Leftrightarrow\ !(n\ \&\ (n - 1)) = 1$
	\item $-a =\ \sim a + 1$
	\item $(4i)\ \oplus\ (4i + 1)\ \oplus\ (4i + 2)\ \oplus\ (4i + 3) = 0$
\end{itemize}
\vspace{-\topsep}
\end{multicols}
\begin{itemize}
	\item Iterating over all subsets of a set and iterating over all submasks of a mask:
	\includecpp{mask.h}
\end{itemize}

\subsection{Combinatorics}
\subsubsection{Catalan numbers}
\[ C_n = \frac{1}{n + 1} {2n \choose n} = \frac{(2n)!}{n!(n+1)!}\]
\[ C_{n + 1} = \displaystyle\sum_{i = 0}^{n}C_i C_{n - i},\ C_0 = 1,\ C_n = \frac{4n - 2}{n + 1}C_{n - 1}\]
\begin{itemize}
	\item The first 12 Catalan numbers $(n = 0, 1, 2, \ldots, 12)$: 
	\[ C_n = 1, 1, 2, 5, 14, 42, 132, 429, 1430, 4862, 16796, 58786 \]
	\item Applications of Catalan numbers:
	\begin{itemize}
		\item difference binary search trees with $n$ vertices from 1 to $n$. 
		\item rooted binary trees with $n + 1$ leaves (vertices are not numbered).
		\item correct bracket sequence of length $2 * n$.
		\item permutation $[n]$ with no 3-term increasing subsequence (i.e. doesn't exist $i < j < k$ for which $a[i] < a[j] < a[k]$).
		\item ways a convex polygon of $n$ + 2 sides can split into triangles by connecting vertices. 
	\end{itemize}
\end{itemize}
% TODO: Catalan convolution

\subsubsection{Stirling numbers of the second kind}
Partitions of $n$ distinct elements into exactly $k$ non-empty groups.
\[ S(n, k) = S(n - 1, k - 1) + kS(n - 1, k)\]
\[ S(n, 1) = S(n, n) = 1 \]
\[ S(n, k) = \frac{1}{k!} \sum\limits_{i = 0}^{k} (-1)^{k - i} {k \choose i} i^n\]

\subsubsection{Derangements}
Permutation of the elements of a set, such that no element appears in its original position (no fixied point). Recursive formulas:
\[ D(n) = (n - 1)[D(n - 1) + D(n - 2)] = nD(n - 1) + (-1)^n\]

\subsection{Pollard's rho algorithm}
	\includecpp{pollard_rho.h}
	